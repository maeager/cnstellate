% ----------------------------------------------------------------
% AMS-LaTeX Paper ************************************************
% **** -----------------------------------------------------------
\documentclass{amsart}
\usepackage{graphicx}
% ----------------------------------------------------------------
\vfuzz2pt % Don't report over-full v-boxes if over-edge is small
\hfuzz2pt % Don't report over-full h-boxes if over-edge is small
% THEOREMS -------------------------------------------------------
\newtheorem{thm}{Theorem}[section]
\newtheorem{cor}[thm]{Corollary}
\newtheorem{lem}[thm]{Lemma}
\newtheorem{prop}[thm]{Proposition}
\theoremstyle{definition}
\newtheorem{defn}[thm]{Definition}
\theoremstyle{remark}
\newtheorem{rem}[thm]{Remark}
\numberwithin{equation}{section}
% MATH -----------------------------------------------------------
\newcommand{\norm}[1]{\left\Vert#1\right\Vert}
\newcommand{\abs}[1]{\left\vert#1\right\vert}
\newcommand{\set}[1]{\left\{#1\right\}}
\newcommand{\Real}{\mathbb R}
\newcommand{\eps}{\varepsilon}
\newcommand{\To}{\longrightarrow}
\newcommand{\BX}{\mathbf{B}(X)}
\newcommand{\A}{\mathcal{A}}
% ----------------------------------------------------------------
\begin{document}

\title[]{CN Network construction and optimisation}%
\author{Michael Eager}%
\address{}%
\email{meager@bionicear.org}%

\thanks{}%
\subjclass{}%
\keywords{}%

%\date{}%
%\dedicatory{}%
%\commby{}%

\maketitle
% ----------------------------------------------------------------
\section{Golgi Cells - rate model}

Due to the unavailability of sufficient data regarding \emph{in vivo} golgi cell
responses, we have decided to simulate the response of golgi cells using inputs
from the Auditory Model filterbanks rather than simulating the neural membrane.

The golgi cell template is similar to the previous CNcell template, so that
minimal changes need to be made to other sections of code.

Each template consists of a spike generator, \emph{s}, and vector objects
representing the instantaneous rate, the spike times and accumulated spike times
of the golgi cell. Parameters identifying each cell include the \emph{channel}
number, CF and bandwidth of ANF input (actually the variance of the weight each
auditory filter channel contributes to the firing of the cell).  The
instantaneous rate vector is derived by

\end{document}
% ----------------------------------------------------------------
